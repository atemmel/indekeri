\documentclass[a4paper, titlepage,12pt]{article}
\usepackage[margin=3.7cm]{geometry}
\usepackage[utf8]{inputenc}
\usepackage[T1]{fontenc}
\usepackage[swedish,english]{babel}
\usepackage{csquotes}
\usepackage[hyphens]{url}
\usepackage{amsmath,amssymb,amsthm, amsfonts}
\usepackage[backend=biber,citestyle=ieee]{biblatex}
\usepackage[yyyymmdd]{datetime}

\title{Inlämmningsuppgift 1}

\author{Adam Temmel (adte1700)}

\begin{document}
	\maketitle

	\section*{Uppgift 1 - Självkostnadskalkyl}
		AB Frys och Frust kan tillverka artiklarna A, Beller C, vilka alla kräver samma slag av material, arbete och maskiner. AB Frys och Frust brukar kalkylera sina produkter efter en kalkyltabellmedföljande utseende.Per år producerar företaget 3 000 st A, 2000 st B, och 1 000 st C.
		\begin{center}
			\begin{tabular}{ |c|c|c|c|c| }
				\hline
				& \textbf{A} & \textbf{B} & \textbf{C} & \textbf{Total} \\
				\hline
				Direkta materialkostanden per st. & 8:- & 10:- & 12:- & 56 000:- \\
				\hline
				MO-pålägg & 25\% & 25\% & 25\% & \\
				\hline
				Direkta löner per st. & 9:- & 9:- & 10:- & 55 000:-\\
				\hline
				TO-pålägg & 60\% & 60\% & 60\% & \\
				\hline
				\textbf{Summa tillverkningskostnader} & \textbf{?} & \textbf{?} & \textbf{?} & \\
				\hline
				AO-pålägg & 40\% & 40\% & 50\% & \\
				\hline
				\textbf{Självkostnad per st.} & \textbf{?} & \textbf{?} & \textbf{?} & \\
				\hline
			\end{tabular}
		\end{center}
		Färdigställ tabellen och beräknakostnaden/styck samt täckningsbidraget.

		\begin{center}
			\begin{tabular}{|c|c|c|c|}
				\hline
				& A & B & C \\
				\hline
				Direkt materialkostnad (dM) & 8:- & 10:- & 12:- \\
				\hline
				Direkta löner per st(dL) & 9:- & 9:- & 10:- \\
				\hline
				Materialkostnad (MO) & 2:- & 2.5:- & 3:- \\
				fördelas som (MO / dM) & $8 \cdot 0.25$ & $10 \cdot 0.25$ & $12 \cdot 0.25$ \\
				$(\frac{14k}{ 56k}) = 0.25$ & & & \\
				\hline
				Tillverkningskostnad (TO) & 5.4:- & 5.4:- & 6:- \\
				fördelas som (TO / dL) & $9 \cdot 0.6$ & $9 \cdot 0.6$ & $10 \cdot 0.6$ \\
				\hline
				Totala tillverkningskostnader (TVK) & 24.4 & 26.9 & 29.4 \\
				är då summan av alla kostnader (dM + dL + MO + TO) & & & \\
				\hline
			\end{tabular}
		\end{center}

		\begin{center}
			\begin{tabular}{|c|c|c|c|}
				\hline
				Administrationskostnader (AO) & 9.76 & 10.76 & 15.5 \\
				fördelas som (AO / TvK) & $24.4 \cdot 0.4$ & $26.9 \cdot 0.4$ & $31 \cdot 0.5$ \\
				$(\frac{63 200}{158 000}) = 0.4$ & & & \\
				\hline
				Självkostnad per st. & 34.16 & 37.66 & 46.5 \\
				är TvK + AO & 24.4 + 9.76 & 26.9 + 10.76 & 31 + 15.5 \\
				\hline
				Täckningsbidrag (TB) & 175200 & 14680 & 3500 \\
				beskrivs som intäkter - rörliga kostnader& $3000 \cdot 40$ & $2000 \cdot 45$ & $1000 \cdot 50$\\
				(St. tillverkad $\cdot$ Intäkt/st.) & $ - 3000 \cdot 34.16$ & $-2000 \cdot 37.66$ & $-1000 \cdot 46.5$ \\
				- (St. tillverkad $\cdot$ Självkostnad/st.) & & & \\
				\hline
			\end{tabular}
		\end{center}

		Slutgilitga tabellen blir då:

		\begin{center}
			\begin{tabular}{|c|c|c|c|c|}
				\hline
				& \textbf{A} & \textbf{B} & \textbf{C} & \textbf{Total} \\
				\hline
				Direkta materialkostanden per st. & 8:- & 10:- & 12:- & 56 000:- \\
				\hline
				MO-pålägg & 25\% & 25\% & 25\% & \\
				\hline
				Direkta löner per st. & 9:- & 9:- & 10:- & 55 000:-\\
				\hline
				TO-pålägg & 60\% & 60\% & 60\% & \\
				\hline
				\textbf{Summa tillverkningskostnader} & \textbf{24.4:-} & \textbf{26.9:-} & \textbf{31:-} & \textbf{157 400:-}\\
				\hline
				AO-pålägg & 40\% & 40\% & 50\% & \\
				\hline
				\textbf{Självkostnad per st.} & \textbf{34.16:-} & \textbf{37.66:-} & \textbf{46.5:-} & \textbf{221 200:-}\\
				\hline
				Intäkt per st. & 40:- & 45:- & 50:- & \\
				\hline
				\textbf{TB} & \textbf{17 520:-} & \textbf{14 680:-} & \textbf{3 500:-} & \\
				\hline
			\end{tabular}
		\end{center}

	\section*{Uppgift 2 - ABC kalkyl}
	Antag att tillverkningsomkostnaderns (TO) uppgår till 600 000kr per åroch antalet arbetstimmar i produktionen är 4000 timmar så att TO-pålägget blir 600 tkr / 4000 tim = 150 kr / tim. Antag att den direkta lönekostnaden är 100kr/tim för alla produkter, samt att AB Frys och Frust uppskattar att: 60\% av TO orsakas av kostnadsdrivaren direkt arbetstid. 20\% av TO orsakas av kostnadsdrivaren antal tillverkningsorder 20\% av TO orsakas av kostnadsdrivaren antal artikelnummer.
		\begin{center}
			\begin{tabular}{|c|c|c|c|c|}
				\hline
				& A & B & C & Totalt \\
				\hline
				Direkt arbetstid & 0.2 tim/st & 0.3 tim/st & 0.8 tim/st & 4000 tim/år \\
				\hline
				Tillverkningsorder & 200 per år & 200 per år & 150 per år & 1000 per år \\
				\hline
				Artikelnummer & 1 st & 1 st & 1 st & 25 st \\
				\hline
				Årsantal & 3000 & 2000 & 1000 & \\
				\hline
			\end{tabular}
		\end{center}
		Beräkna kostnad/styck för produkterna A, B, och C enligt modellen akvitivetsbaserad kostnadskalkyl.
		\begin{center}
			\begin{tabular}{|c|c|c|c|c|}
				\hline
				& \textbf{A} & \textbf{B} & \textbf{C} & \textbf{Totalt} \\
				\hline
				Direkt arbetstid & 12 min & 18 min & 48 min & 4000tim/år \\
				\hline
				Tillverkningsorder & 200 per år & 200 per år & 150 per år & 1000 per år \\
				\hline
				Antal komponenter & 1 & 1 & 1 & 25 \\
				\hline
				Årsantal & 3000st & 2000st & 1000st & 6000st \\
				\hline
				\textbf{Kostnadsdrivare} & & \textbf{Kostnad/enhet} & &\\
				\hline
				Arbetstimmar & $(0.6\cdot0.6)Mkr/4000h$ & 90kr/h & & \\
				\hline
				Tillverkningsorder & $(0.2\cdot0.6)Mkr/1000st$ & 120kr/order & & \\
				\hline
				\emph{n} Komponenter & $(0.2\cdot0.6)Mkr/25st$ & 4800kr/\emph{n} och år & & \\
				\hline
			\end{tabular}
		\end{center}

		\begin{center}
			\begin{tabular}{|c|c|}
				\hline
				\textbf{A} & \textbf{Kostnad/st} \\
				\hline
				dL & $0.2h \cdot 100kr = 20kr$ \\
				\hline
				Arbetstimme & $0.2h \cdot 90kr = 18kr$ \\
				\hline
				Order & $200st \cdot 120kr / 3000 = 8kr$ \\
				\hline
				Komponenter & $1 \cdot 4800 / 25 = 192kr$ \\
				\hline
				Summa & $20 + 18 + 8 + 192 = 238$ \\
				\hline
			\end{tabular}
		\end{center}

		\begin{center}
			\begin{tabular}{|c|c|}
				\hline
				\textbf{B} & \textbf{Kostnad/st} \\
				\hline
				dL & $0.3h \cdot 100kr = 30kr$ \\
				\hline
				Arbetstimme & $0.3h \cdot 90kr = 27kr$ \\
				\hline
				Order & $200st \cdot 120kr / 2000 = 12kr$ \\
				\hline
				Komponenter & $1 \cdot 4800 / 25 = 192kr$ \\
				\hline
				Summa & $30 + 27 + 12 + 192 = 261kr$ \\
				\hline
			\end{tabular}
		\end{center}

		\begin{center}
			\begin{tabular}{|c|c|}
				\hline
				\textbf{C} & \textbf{Kostnad/st} \\
				\hline
				dL & $0.8h \cdot 100kr = 80kr$ \\
				\hline
				Arbetstimme & $0.8h \cdot 90kr = 72kr$ \\
				\hline
				Order & $150st \cdot 120kr / 1000 = 18kr$ \\
				\hline
				Komponenter & $1 \cdot 4800 / 25 = 192kr$ \\
				\hline
				Summa & $80 + 72 + 18 + 192 = 362kr$ \\
				\hline
			\end{tabular}
		\end{center}

	\section*{Uppgift 3 - Nuvärdeskalkyl}
		\subsection*{a)} AB Frys och Frust har två olika kapitalkällor $K_1$ och $K_2$. $K_1$ är ett banklån på värdet 2 miljoner kronor och ett återbäringskrav på 4\%, och $K_2$ är riskkapital med ett marknadsvärde på 1 miljon kronor och ett återbäringskrav på 12\%. Beräknaden viktade kostnaden av kapital (WACC) och använd det som kalkylränta.\\

$K_1 = 2 000 000kr$ med $4\%$ avkastningskrav\\

$K_2 = 1 000 000kr$ med $12\%$ avkastningskrav\\

$K = K_1 + K_2$\\

$r_1 = 4\%$\\

$r_2 = 12\%$\\

$WACC = \frac{K_1}{K} \cdot r_1 + \frac{K_2}{K} \cdot r_2 = \frac{200000000}{3000000} \cdot 0.12 = 0.0666667 = 6.67\%$\\

		\subsection*{b)} AB Frys och Frust ska nu investera åttamiljoner i en eller två nya produkter, produkt \textbf{D}, \textbf{E}, eller \textbf{F}. Du beaktar kassaflöden över fem års tid och kommer fram till tabellen nedan där respektive projekts årliga kassaflöden ges för fem år. Vilket/vilka projekt rekommenderas enligt nunettovärdesmetodengivet att du använder ditt WACC som diskonteringsfaktor? Alla tre investeringar har samma grundinvestering \emph{G}, och alla värden i tabellen är i tkr.
			\begin{center}
				\begin{tabular}{|c|c|c|c|c|c|c|}
					\hline
					Produkt & \emph{G} & År 1 & År 2 & År 3 & År 4 & År 5 \\
					\hline
					D & -400 & 100 & 110 & 120 & 130 & 140 \\
					\hline
					E & -400 & 80 & 85 & 130 & 160 & 180 \\
					\hline
					F & -400 & 105 & 110 & 115 & 125 & 135 \\
					\hline
				\end{tabular}
			\end{center}

			$WACC = 6.67\%$\\

			$NNUV = \sum_{i=1}^{n} \frac{x_i}{(1 + r)^n} - G$ \\

			$N$ är den ekonomiska livslängden, $r$ är kalkylräntan, $x_i$ är kassaflödet år $i$ och $G$ är grundinvesteringen. Vissa värden kan man nu stoppa in.\\

			$NNUV = \sum_{i=1}^{5} \frac{x_i}{(1+0.067)^5} - 400$\\

			$D = \frac{100}{1.383} + \frac{110}{1.383} + \frac{120}{1.383} + \frac{130}{1.383} + \frac{140}{1.383} - 400 = 72.307 + 79.537 + 86.768 + 93.999 + 101.229 - 400 = 33.84$ tusen kr\\

			$E = \frac{80}{1.383} + \frac{85}{1.383} + \frac{130}{1.383} + \frac{160}{1.383} + \frac{180}{1.383} - 400 = 59.147$ tusen kr\\

			$F = \frac{105}{1.383} + \frac{110}{1.383} + \frac{115}{1.383} + \frac{125}{1.383} + \frac{135}{1.383} - 400 = 26.609$ tusen kr\\

		\subsection*{c)} Vad har investeringarna i \textbf{D}, \textbf{E}, och \textbf{F} för internränta?
		
			D: $-400 + \frac{100}{1 + R} + \frac{110}{(1 + R)^2} + \frac{120}{(1 + R)^3} + \frac{130}{(1 + R)^4} + \frac{140}{(1 + R)^5} = 0$\\

			$D = 0.143 = 14.3\%$\\

			E: $-400 + \frac{80}{1 + R} + \frac{85}{(1 + R)^2} + \frac{130}{(1 + R)^3} + \frac{160}{(1 + R)^4} + \frac{180}{(1 + R)^5} = 0$\\

			$E = 0.150 = 15\%$\\

			F: $-400 + \frac{105}{1 + R} + \frac{110}{(1 + R)^2} + \frac{115}{(1 + R)^3} + \frac{125}{(1 + R)^4} + \frac{135}{(1 + R)^5} = 0$\\

			$F = 0.138 = 13.8\%$\\

		\subsection*{d)} Vilken investering är mest effektiv enligt nuvärdekvoten?\\

		Nuvärdeskvoten $NUV$ blir $\frac{NNUV}{G}$ så:\\

		$D = \frac{33.84}{400} = 0.0846$ tusen kr $= 84.6$kr\\

		$E = \frac{59.147}{400} = 0.1478675$ tusen kr $= 14.8$kr\\

		$F = \frac{26.609}{400} = 0.0665225$ tusen kr $= 66.5$kr\\

		$D$ är bäst, då högre = bättre.\\

		\subsection*{e)} Vad är annuiteten för \textbf{D}, \textbf{E}, respektive \textbf{F}?\\

		$\alpha = \frac{r}{1 + (1 + r)^-n} = \frac{0.067}{1-(1+0.067)^-5} = 0.242$\\

		$ANN = NNUV \cdot \alpha$\\

		D: $33.84 \cdot 0.242 = 8.18928$ tusen kr $\approx 8200$kr\\

		E: $59.147 \cdot 0.242 = 14.313574$ tusen kr $\approx 14 300$ kr\\

		F: $26.609 \cdot 0.242 = 26.609$ tusen kr $\approx 26 600$ kr\\
\end{document}
